\documentclass[../masterarbeit.tex]{subfiles}

\section{Related Work}
In the context of avalanche risk prediction, several studies have already been conducted. Machine learning methods have been applied on this purpose. Data sets of avalanche events from smaller mountain areas were used as case studies. In the alps a dataset of avalanche events from the area around Davos, Swizerland over the last 13 years is one Example for these case studies. The study about the data from Davos aimed to predict an entire winter season of avalanche days.
To get the Meteorological Data for their study the Team form the SLF, Swizerland combined the data of snow avalanche events with those of an automatic weather station and the simulated snowpack properties, like new snow depth, liquid water content, Stability indices, critical crack length,  and the hand hardness,  of the SNOWPACK model. A random forest model was then trained on these merged data. One finding of the resulting study is that the predictions without the snowpack factors are just as good as those without this additional data. Furthermore, they concluded that their prediction attempts were severely limited by the use of inaccurate, biased avalanche-related data and by the fact that the spatial scale was too large for their models. They came to the result that forecasting on a small spatial scale using only one avalanche warning system could work well and could be a good aid for avalanche hazard forecasting services. \autocite{Harvey:2016}.
In another study, avalanche hazard maps were attempted using predictions from machine learning methods.  The space around the watershed of Karaj, Iran was used in this paper. The machine learning methods used in this study were SVM and MDA.Also for this purpose, various meteorological data were brought in and used to train the algorithms. From the study, it was found that avalanches seem to slide out mainly in the vicinity of streams. In addition, MDA performed better in predicting avalanche danger compared to SVM. Both methods produced results with an accuracy of 0.83 for the SVM and 0.85 for the MDA. \autocites{Bahram:2019}.



\end{document}