\documentclass[../masterarbeit.tex]{subfiles}
\begin{document}
	



\section{Introduction}
Avalanches are among the most dangerous natural disasters \textcite{Lawine:2019}. They threaten people and environment in seasonally snow-covered mountain regions such as the alps. In 2021, 265 recorded avalanches (recoreded on Lawis) in Austria killed 17 people and injured 5 others. A total of 41 people were involved in the events \textcite[]{Lawis:2022}. The occurrence of snow avalanches is steadily increasing due to climate change \textcite{Martin:2001} \textcite[]{Tiwari:2021} \textcite{Bahram:2019}.  In order to enable preventive measures to combat avalanches, a good assessment of the impending danger is necessary. The complexity of this task has already been described in several studies. It comes from the fact that there are many potentially influancial parameters, of which in most scenarios only a few are available and and they change significantly even for small geographical differences. For example, the estimation of avalanche danger levels is mostly based on weather data, which are also predicted, results of snowpack models and data on snow instability. In addition, information on the terrain is often used for these features \textcite{Bahram:2019}. Avalanches have been artificially triggered by explosions for a long time. For this preventive fight against snow avalanche accidents, automatic and semi-automatic snow avalanche detection systems have been developed. Of these, infrasound-based systems are the most promising, in the early detection of slow avalanches, for example ice avalanches \textcite{THURING201560} \textcite{Lawine:2019}. In addition, these systems are used to detect as many avalanches as possible. Many avalanches have not been recorded, which means that the accuracy of predictions made by automatic systems is compromised. Geographic Information Systems (GIS), Hierarchy Process Methods (AHP), and Remote Sensing (RS) are used together for this purpose to assess and assess avalanche hazards. With the help of these systems avalanche hazards can be detected, but they have a high error rate. Therefore, predicting the risk of avalanches on selected slopes would be a great improvement in the precision of preventing avalanches by blasting them. Machine learning\footnote{cf. chapter 3.2 of this thesis} is already being used for various real-world problems \textcite[]{SUBASI202091} and it also has been revealed that some machine learning models can achieve good results in predicting natural disasters and explicit avalanches \textcite[]{Martin:2001} \textcite[]{Tiwari:2021} \textcite{Bahram:2019}. 
Various systems also use the Nearest Neighbor\footnote{"The k-nearest neighbors algorithm, also known as KNN or k-NN, is a non-parametric, supervised learning classifier, which uses proximity to make classifications or predictions about the grouping of an individual data point. While it can be used for either regression or classification problems, it is typically used as a classification algorithm, working off the assumption that similar points can be found near one another." \autocite[]{ibm-Nearest-Neighbors:2022} } method for their predictions \textcite[]{Pozdnoukhov:2008}.
Avalanche warning services could receive promising forecasts for the day and later for future days in advance and make them available to the public. This leads to a significant increase in safety for backcountry skiers and other winter mountain sports enthusiasts.
Backcountry skiing is a sport in which the athletes are skiing unmarked or unguarded areas within or outside the boundaries of a ski resort. For this, they either climb the mountain with touring skis or are brought up by helicopter or snowmobile. Having good predictions for the avalanche hazard, makes it easier and safer to plan these tours. This has led to the author's personal interest for this topic and the resulting possibilities to get high precision predictions of avalanches hazard for topographical defined explizit mountain slopes by the use of machine learning models.
Since there are people who work in avalanche-prone areas, the forecast for defined mountain slopes would also improve their work safety.  For example, some employees of the editor of the data work in the area of the Glockner Group. The data provided for this work is collected purely for the purpose of improving work safety for this avalanche-prone area.
\newline
\newline
The aim of this master thesis is to predict snow avalanches for topographical explicit defined mountain slopes. In addition, another task is to find out which machine learning models are best suited for this task and predicts the snow avalanches with the best quality. As well as which factors contain the most meaningful informationS and have impact on the prediction quality of the machine learning models.  This is done by using topographical data (e.g. slope, slope orientation by cardinal directions, altitude) in conjunction with meteorological data for the day to be predicted (e.g. weather, temperature, wind direction, wind speed, new snow), which will be also used retrospectively two to four days in the past for selected data. Also Snow pack related data (e.g. snow depth, snow temperature, snow sinking depth) is used in combination with the meteorological data. Thus, the study is intended to provide an answer to the questions, "Does the meteorological data contain enough information to apply machine learning to predict snow avalanches for topographically defined mountain slopes?" and "With what quality can snow avalanches be predicted from meteorological data for topographically defined mountain slopes, in the specific context of this case study in Kaprun/Mooserboden?". \\~\\


This master thesis is structured as follows. chapter 2 describes similar work and work related to this thesis. In the following 3rd chapter the methods used for the thesis are described. There are three overlapping subchapters. 3.1 describes the methods used for data preparation. 3.2 on the methods, which were used for the feature selection. Section 3.3 explains the three machine learning algorithms used for the study and section 3.4 the evalutation metrics used to validate the performance of the three machine learning models.
Chapter 4 describes the results of the study. The results chapter is again divided into three main subsections. The first subsection describes the origin of the data, the composition of the data set and the resulting data set. This is followed by a two-step description of the feature selection process. The last major part of the Results chapter deals with the training and evaluation of the three machine learning models. 
In chapter 5 the results are discussed in comparison with the results from some of the studies from chapter 2. Following the Discussion Chapter, the Limitations that emerged based on the study on the topic are explained in Chapter 6. The outlook for possible future work is then described in the 7th Outlook chapter. As the last chapter, a summary description of the results of this master thesis is presented in the Conclusion. 

















\end{document}                 