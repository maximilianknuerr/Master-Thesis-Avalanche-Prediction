\documentclass[../masterarbeit.tex]{subfiles}





\section{Introduction}
Avalanches are among the most dangerous natural disasters \autocite{Lawine:2019}. They threaten people and environment in seasonally snow-covered mountain regions such as the alps. In 2021, 265 recorded avalanches (recoreded on Lawis) in Austria killed 17 people and injured 5 others. A total of 41 people were involved in the events \autocite[]{Lawis:2022}. The occurrence of snow avalanches is steadily increasing due to climate change \autocite{Martin:2001} \autocite[]{Tiwari:2021} \autocite{Bahram:2019}.  In order to enable preventive measures to combat avalanches, a good assessment of the impending danger is necessary. The complexity of this task has already been described in several studies. It comes from the fact that there are many potentially influancial parameters, of which in most scenarios only a few are available and and they change significantly even for small geographical differences. For example, the estimation of avalanche danger levels is mostly based on weather data, which are also predicted, results of snowpack models and data on snow instability. In addition, information on the terrain is often used for these features \autocite{Bahram:2019}. Avalanches have been artificially triggered by explosions for a long time. For this preventive fight against snow avalanche accidents, automatic and semi-automatic snow avalanche detection systems have been developed.Of these, infrasound-based systems are the most promising, in the early detection of slow avalanches, for example ice avalanches \autocite{THURING201560} \autocite{Lawine:2019}. In addition, these systems are used to detect as many avalanches as possible. Many avalanches have not been recorded, which means that the accuracy of predictions made by automatic systems is compromised. Geographic Information Systems (GIS), Hierarchy Process Methods (AHP), and Remote Sensing (RS) are used together for this purpose to assess and assess avalanche hazards. with the help of these systems avalanche hazards can be detected, but they have a high error rate. Therefore, predicting the risk of avalanches on selected slopes would be a great improvement in the precision of preventing avalanches by blasting them.  It has been revealed that some machine learning can achieve good results in predicting natural disasters and explicit avalanches \autocite[]{Martin:2001} \autocite[]{Tiwari:2021} \autocite{Bahram:2019}. Avalanche warning services could receive promising forecasts for the following days in advance and make them available to the public. This leads to a significant increase in safety for backcountry skiers and other winter mountain sports enthusiasts.
\newline
Backcountry skiing is a sport in which the athletes are skiing unmarked or unguarded areas within or outside the boundaries of a ski resort. For this, they either climb the mountain with touring skis or are brought up by helicopter or snowmobile. Having good predictions for the avalanche hazard, makes it easier and safer to plan these tours. This has led to the author's personal interest for this topic and the resulting possibilities to get high precision predictions of avalanches hazard for explizit Slopes by the use of machine learning models.
\newline
\newline
The aim of the master thesis is to predict the probability of avalanches from explicit slopes. These predictions should be made for a certain number of days in the future, for example for one week in the future. This is done by using terrain related data (e.g. slope, slope orientation by cardinal directions or proximity to rivers and streams) in conjunction with meteorological data for the day to be predicted (e.g. weather, temperature, snow depth, wind direction), wich will be also used retrospective. Thus, the study is intended to provide an answer to the questions, "What data features are needed to predict a day-accurate avalanche probability for explicit slopes by the use of machine learning methods?" and "How many days in advance can these predictions be determined?". 





\end{document}