\documentclass[12pt,a4paper]{article}
\usepackage[includeheadfoot,margin=2.5cm]{geometry}

\usepackage{times}
\usepackage[utf8]{inputenc}
\usepackage{listings}

\usepackage{csquotes}
\usepackage{abbrevs}
\usepackage[backend=bibtex, style=ieee, citestyle=ieee]{biblatex}
\bibliography{Bibliography}

\usepackage[english,ngerman]{babel}

\newabbrev{\authorid}{2010695005}
\newabbrev{\authorname}{Maximilian Knürr}
\newabbrev{\authormail}{mknuerr.mmt-b2015@fh-salzburg.ac.at}
\newabbrev{\exposedate}{02. January 2022}
\newabbrev{\titlename}{The prediction of snow avalanches based on selected meteorological factors for topographically defined mountain slopes using machine learning models. A case study in Kaprun/Mooserboden}
\newabbrev{\supervisor}{Supervisor}
\newabbrev{\address}{FH Salzburg}
\newabbrev{\thesisdate}{Salzburg, Austria, 30.September 2022}


\title{\titlename}

\author{ \authorname\\ \scriptsize \authormail \\ \scriptsize \address }
\author{ \authorid\\ \scriptsize \address }


\date{\exposedate}


\begin{document}
\selectlanguage{english}



\maketitle

\tableofcontents
\newpage



\section{Daten}

\subsection{Herkunft der Daten}
Der Lawinenwarndienst des Energieunternehmens Verbund AG in Kaprun, Salzburg stellt die für die durchführung dieser Arbeit notwendigen Daten bereit. Die Daten bestehen aus mehreren Tabellen, die im Kontext von 39 Lawinenstrichen in der Umgebung des Speichersees Mooserboden, Kaprun stehen. Die für diese Arbeit relevanten Tabellen beinhalten Geografische Faktoren der 39 Lawinenstriche, Lawinenabgänge, sowie Wetterdaten im Zeitraum von 1956 bis 2022. Um eine größere Homogenität der Daten zu gewährleisten wurden ausschließlich Daten ab 1992 für diese Arbeit genutzt.

\subsubsection{Wetterdaten}

\subsubsection{Lawinen bezogene Daten}

\subsubsection{Geografische Daten}











\newpage
\addcontentsline{toc}{section}{References}
\printbibliography








\end{document}
